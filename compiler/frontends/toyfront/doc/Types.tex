% Written by:
% * Joachim Trescher
% * Hans de Vreught

\section{Types}

In \Booster\ a {\em type} is specified by an expression in
\Language{\NT{Type}} and represents a set of values.  \Booster\ uses
structural type equivalence: two types are the same if their
definitions become the same when expanded; i.e.\ when all constant
expressions are replaced by their values and all type names are
replaced by their definitions.

Every expression has a statically determined type, which contains
every value that the expression can produce. The type of a designator
is the type of the variable it produces. 

{\em Assignability} and {\em type compatibility} are defined in terms
of a syntactically specified inclusion relation on types. In the
following we discuss type constructors and type inclusion rules in more
detail.

\subsection*{Intrinsic Types}

There are six intrinsic types which are denoted by the following
predeclared identifiers.

\begin{tabular}{ll}
\hlf
\T{BOOLEAN} & the enumeration $\{$ {\sf true, false} $\}$\\
\T{NATURAL} & all natural values represented by the implementation\\
\T{INTEGER} & all integers represented by the implementation\\
\T{REAL}    & all real values represented by the implementation\\
\T{COMPLEX} & all complex values represented by the implementation\\
\T{STRING}  & the elements of \Language{\TC{StringLiteral}}\\
\hlf
\end{tabular}


\subsection*{Shape and view types}

A shape type specifies an indexed collection of component variables,
called the {\em elements} of the shape. A shape type declaration has
the syntactical form:

\begin{frag}
TYPE S = SHAPE \{ E \} OF BaseType;
\end{frag}

\noindent where {\tt E} is a constant expression of type natural and
{\tt BaseType} is a type specification.

The indices are the natural values in the range {\tt [0..E-1]}, where
the value of {\tt E} is determined at compile time and is called the
length of the shape. The elements of a shape all have the same size
and the same type, called the {\em base type}.

An expression of the syntactical form

\begin{frag}
SHAPE \{<E$_{\tt 0}$>  \verb'#' <E$_{\tt 1}$>  \verb'#' \ldots \verb'#'
<E$_{\tt n}$>\} OF <T>; 
\end{frag}

\noindent is understood as an abbreviation of 

\begin{frag}
SHAPE \{<E$_{\tt 0}$>\} OF\\
\> SHAPE \{<E$_{\tt 1}$>\} OF\\
\>\> \dots\\
\>\>\> SHAPE \{<E$_{\tt n}$>\} OF <T>;
\end{frag}

\noindent The form of a shape is the sequence of its lengths in each
dimension. More precisely, the form of a shape is its length followed
by the form of its base type. The length of the form of a shape is
called its {\em rank}.

In the declaration of formal parameters of functions and procedures
\Booster\ allows to specify the length of a dimension by a
\texttt{CONST <Id>} declaration. The corresponding
dimension of such a parameter is arbitrary but does not change during
the execution of a procedure call. The {\em generic} constant
\texttt{<Id>} may be used as a constant in the body of the
corresponding function or procedure. Each function or procedure call
initializes the generic constant \texttt{<Id>} with the length of the
corresponding dimension of the actual parameter.

A view type is a ``virtual'' shape type whose rank and (intrinsic)
base type are determined by a view type specification. Unlike shapes,
a view type specification may specify the length of one or more
dimensions to be arbitrary (and possibly changing dynamically) using
the operator \T{*}. Informally, views may be considered as bounded set
of references which point to elements (whose type is an intrinsic
type) of a single variable, thus ``viewing'' the elements of this
variable through a different indexing scheme. However, the declaration
of a view will not cause the allocation of computer memory. Therefore,
during the compilation process views will be eliminated by replacing
accesses to them by calls to the appropriate access function.

A view type declaration has the syntactical form

\begin{frag}
TYPE V = VIEW \{<E$_{\tt 0}$>  \verb'#' <E$_{\tt 1}$>  \verb'#'
\ldots \verb'#'  <E$_{\tt n}$>\} OF IntrinsicType; 
\end{frag} 

\noindent where the form of \textsf{V} is defined by the expressions
{\tt <E$_{\tt 0}$>} \ldots {\tt <E$_{\tt n}$>}. And {\tt
IntrinsicType} is an intrinsic type.

\subsection*{Defined-by and Uses relations}

The type declarations visible in a scope $S$ define two relations on
type identifiers. The {\em defined-by} relation ``$\succ$'' is used to
track down type equivalences, and is defined as follows:

A type identifier $Id_2$ is defined by a type identifier $Id_1$, or
$Id_2 \succ Id_1$, iff

\begin{itemize}

\item there is a type declaration of the form {\tt <Id$_{\tt 2}$> =
<Id$_{\tt 1}$>}  visible in $S$, or

\item there exists a type identifier $Id_3$ in $S$ such that,
\(
Id_2 \succ Id_3 \wedge  Id_3 \succ Id_1
\)
\end{itemize}


The {\em uses} relation ``$\gg$'' is used to check which type is used
in the type declaration of an other type identifier. A type identifier
$Id_1$ uses a type identifier $Id_2$, or $Id_1 \gg Id_2$, iff

\begin{itemize}

\item in $S$ a type declaration of the following form is visible

\begin{frag}
<Id$_{\tt 1}$> =
SHAPE <CardList> OF <Id$_{\tt 2}$>;
\end{frag}

\item there exists a type identifier $Id_3$ such that,
\(
Id_1 \gg Id_3 \wedge  Id_3 \gg Id_2
\)
\end{itemize}

\noindent The identifiers of a given scope and name space must be
well-defined. Therefore, we require the ordering induced by the union
of the relations $\succ$ and $\gg$ to be a strict partial ordering
(i.e.\ there are no cycles in the defined-by and uses relations). The
minimal elements of this induced ordering are intrinsic types (i.e.\
all \Booster\ types are defined in terms of intrinsic types and type
constructors). Moreover, there is no identifier $Id: Types.tex,v 1.1.1.1 1995/10/17 13:34:01 leo Exp $ such that \(
\mbox{\em IntrinsicType} \gg Id \; \vee \; \mbox{\em IntrinsicType}
\succ Id \) for any intrinsic type (i.e.\ intrinsic types may not be
redefined).

We define the set of all types {\em used} by a variable declaration as
the set of type identifiers {\small \tt <Id'>} such that

\begin{itemize}

\item the variable declaration has the form {\tt "<IdList> : SHAPE
<CardList> OF <Id>;" }

\item the variable declaration has the form { \tt "<IdList> : <Id>;" }
\item there exists an {\small \tt <Id>} such that {\small \tt <Id>}
$\gg$ {\small \tt <Id'>} and one of the above conditions hold for
{\small \tt <Id'>}.

\end{itemize}

\subsection*{Relations on types}

\noindent In order to define some relations on types we first
introduce two functions on \Booster\ types. The function {\em form}
computes the form of a member of a type and represents it as a
sequence. In the result of the function \form\ a definite length is
represented by a natural number, the generic constants in the
declaration of a formal parameter are represented by a {\em generic
variable}, and an indefinite dimension of a view type is represented
by a {\em free variable}. The base type function \bt\ yields the base
type of a type identifier by first expanding a type (i.e.\ recursively
replacing all identifiers by their definition) and then returning the
base type of the expanded type. Thus the result of an application of
the function \bt\ is always an intrinsic type. More formally:

Let \genvar\ and \freevar\ be infinite sets of generic and free
variables, respectively, such that $\freevar \cap \genvar =
\emptyset$. Let $\Forms ~ \widehat{=} ~ \mbox{\bf seq}(\mbox{\tt
NATURAL} \cup \genvar \cup \freevar)$, and let the operator $\concat$
denote the concatenation of sequences. In the following definitions
the identifiers $F$ and $G$ denote a fresh free or generic variable,
respecitvely, $C$ and $C'$ denote an element of
$\Language{\NT{CardinalityList}}$, $c$ denotes an element of
$\Language{\NT{Cardinality}}$, $E$ denotes an element of
$\Language{\NT{Expressions}}$, and {\em Id} denotes an element of
$\Language{\TC{Identifier}}$. Given the set \TypeDefSet\ we define:

\[
\begin{array}{rcl}
\multicolumn{3}{c}{
\form: \Language{\NT{Type}} \rightarrow \Forms}\\
\hlf
\form(T) & \widehat{=} &
\left\{\begin{array}{ll}
\langle \rangle	
        & \mbox{if } T = \T{STRING} \vee\\
	& \mbox{if } T = \T{BOOLEAN} \vee\\
	& \mbox{if } T = \T{NATURAL} \vee\\
	& \mbox{if } T = \T{INTEGER} \vee\\
	& \mbox{if } T = \T{REAL} \vee\\
	& \mbox{if } T = \T{COMPLEX} \vee\\
\form_c(C)\concat\form(T')
	& \mbox{if } T = \T{SHAPE}\ \T{\{}\ C\ \T{\}}\ \T{OF}\ T' \\
\form_c(C)\concat\form(T')
	& \mbox{if } T = \T{VIEW}\ \T{\{}\ C\ \T{\}}\ \T{OF}\ T' \\
\form(T')
        & \mbox{if } T' \in \Language{\NT{Type}}\ \mbox{for some } (T,
	T') \in \TypeDefSet \\
\end{array}
\right.\\
\end{array}
\]

\noindent
The cardinality function $\form_{c}: \Language{\NT{CardinalityList}} 
\rightarrow \Forms$, auxiliar to \form, is defined as follows:

\[
\begin{array}{rcl}
form_{c}(C) & \widehat{=} & \left\{
\begin{array}{ll}
\langle v(E)\rangle
	& \mbox{if } C = Id\ \T{:}\ E \\
	& \mbox{or } C = E \\
\langle F \rangle
	& \mbox{if } C = Id\ \T{*}\\
\langle G \rangle
	& \mbox{if } C = \T{CONST}\ Id \\
\form_c(c)\concat\form_c(C')
	& \mbox{if } C = c\ \T{\#}\ C' \\
\end{array}
\right.\\
\end{array}
\]

\noindent where the function $v$ evaluates to a natural value if its
argument is a constant expression, or returns a free new variable from
\genvar:

\[
\begin{array}{rcl}
\multicolumn{3}{c}{
v: \Language{\NT{Expression}} \rightarrow \mbox{\tt NATURAL} \cup
\genvar}\\
\hlf
v(E) & \widehat{=} & \left\{
\begin{array}{ll}
n 
  & \mbox{if $E$ is a constant expression that evaluates to } n\\
G
  & \mbox{for a fresh free variable $G \in \genvar$ otherwise}
\end{array}
\right.
\end{array}
\]

\noindent We now define the base type function \bt:

\[
\begin{array}{rcl}
\multicolumn{3}{c}{
\bt: \Language{\NT{Types}} \rightarrow \Language{\NT{Types}}}\\
\hlf
\bt(T) = \left\{\begin{array}{ll}
\bt(T') & \mbox{if } T = \T{SHAPE}\ C\ \T{OF}\ T'\\
T'      & \mbox{if } T = \T{VIEW}\ C\ \T{OF}\ T'\\
\bt(T') & \mbox{if } (T,T') \in \TypeDefSet \mathrm{ for\ some } T'\\
T	& \mbox{otherwise}
\end{array}
\right.
\end{array}
\]

\noindent Given the following transitive and symmetric {\em inclusion}
relation on intrinsic types

\begin{quote}
\(
\T{COMPLEX} \sqsupseteq \T{REAL} \sqsupseteq
\T{INTEGER} \sqsupseteq \T{NATURAL}
\)
\end{quote}

\noindent we say two types $T_1$ and $T_2$ are {\em compatible}
$(\approx$) if the base type of $T_1$ includes the base type of $T_2$
and $\form(T_1)$ and $\form(T_2)$ can be made equal by an appropriate
substitution of the occuring free and generic variables. A type $T_1$
{\em includes} $(\sqsupseteq)$ a type $T_2$, if $T_1$ is a
(multi-dimensional) shape of type $T$ and $T$ is compatible to $T_2$. In
the following we define the type compatibility and type inclusion
relations formally.

\[
\begin{array}{rcl}
T_1 \approx T_2 & \widehat{=} 
   & \bt(T_1) \sqsupseteq \bt(T_2) \wedge\\
 & & \mathrm{Let}\ \form(T_1) = \langle e_1,\ldots,e_m\rangle;
 \form(T_2) = \langle d_1,\ldots,d_n\rangle\ \mathrm{then} \\
 & & \tab n = m \wedge \forall i \in \{1,\ldots,n\} \bullet (e_i = d_i \vee\\
 & & \tab \tab (\{e_i, d_i\} \cap \freevar \cup \genvar \neq \emptyset)\\
\hlf
T_1 \sqsupseteq T_2 & \widehat{=} 
   & \bt(T_1) \sqsupseteq \bt(T_2) \wedge\\
 & & \mathrm{Let}\ \form(T_1) = \langle e_1,\ldots,e_m\rangle;
     \form(T_2) = \langle d_1,\ldots,d_n\rangle\ \mathrm{then} \\
 & & \tab \exists i \in \{1,\ldots,m\} \bullet \langle e_i,\ldots,e_m\rangle
     = \langle d_1,\ldots,d_n\rangle
\end{array}
\]
