\section{Definitions}

\Booster\ programs specify computations that act on components
called {\em locations}. {\em Variables\/} are bounded sets of
locations that represent a value determined by the variable's {\em
type}. If a value {\sc v} can be represented by a variable of type
{\sc t}, then we say the value is a {\em member} of {\sc t}, and {\sc
t} {\em contains} the value {\sc v}.

The {\em form} of a variable is the sequence of its length in each
dimension. The form of a zero dimensional variable (i.e.\ scalar) is
the empty sequence.

An {\em expression\/} specifies a computation that produces a value, a
variable, or an enumeration of natural values called a {\em
selection}. Expressions that produce variables are called {\em
designators}. A designator can denote either a variable or the value
of that variable, depending on the context. Expressions that produce
selections are called {\em set expressions}. Expressions whose values
can be determined statically are called {\em constant expressions}.

\subsection*{Notational conventions}

For the rest of this document we adopt the following notational
conventions: 

\begin{itemize}

\item let \NT{NT} be a nonterminal of the \Booster\ grammar, the term
\Language{\NT{NT}} denotes the language defined by the production
rules of the \Booster\ grammar and the start symbol \NT{NT}.

\item For the representation of \Booster\ program fragments we will use
constructs of the extended BNF defined in Chapter \ref{BoosterGrammar}
in an obvious way. Identifiers surrounded by \verb'<' and \verb'>'
denote the language construct suggested by the Booster grammar. For
instance, in the program fragment

\begin{frag}
{[}IMPLEMENTATION{]} MODULE <Id$_{\tt 1}$>;\\
\> <ImplementationUnits>\\
\> [BEGIN <StatementList>]\\
END <Id$_{\tt 2}$>.
\end{frag}

\noindent the parts between brackets are optional. Furthermore, {\small
\tt <Id$_{\tt 1}$>} and {\small \tt <Id$_{\tt 2}$>} are elements of
\Language{\TC{Identifier}}, {\small \tt <ImplementationUnits>} $\in$
\any{\Language{\NT{ImplementationUnit}}}, and {\small \tt
<StatementList>} $\in$ \Language{\NT{StatementList}}.

\end{itemize}

\subsection*{Representation}

The representation of terminal symbols in terms of characters is
defined using the ASCII set. Symbols are identifiers, numbers,
strings, operators, and delimiters. The following lexical rules must
be observed: Blanks and line breaks must not occur within symbols,
they are ignored unless they are essential to separate two consecutive
symbols. Capital and lower-case letters are considered as distinct.

\begin{description}

\item[Identifiers] are symbols (represented by a sequence of letters
and digits whose first character is a letter) declared as a name for a
type, variable, procedure etc.

\item[Numbers] are natural, integer, real, or complex
constants. The type of a natural or  integer constant is the minimal
type to which the value belongs. A real number always contains a
decimal point; optionally it may also contain a decimal scale factor.
The letter ``E'' means ``times ten to the power of.'' A complex number
is represented by a tuple $(r_1, r_2)$ of two real numbers where $r_1$
represents the real part and $r_2$ represents the imaginary part of
the complex number.

\item[Strings] are sequences of characters enclosed in single (')
quote marks.

\item[Operators and delimiters] are the special symbols listed
below. The reserved words consist exclusively of capital letters and
must not be declared as identifiers.

{\small \tt
\begin{tabular}{lllll}
\verb'+' & \verb':='  & AND            & IN        & THEN\\
\verb'-' & \verb'*'   & BEGIN          & INOUT     & TIMES\\
\verb'=' & \verb'<>'  & CONST          & ITER      & TYPE\\
\verb'>=' & \verb'<=' & DEFINITION     & MOD       & VAR\\
\verb'#' & \verb'&'   & DIV            & MODULE    & VIEW\\
\verb'<-' & \verb"//" & DO             & MACHINE   & WHILE\\    
\verb'|' & \verb"'"   & ELSE           & NOT\\	   
\verb'/' & \verb'..'  & ELSIF          & OF\\
\verb',' & \verb'.'   & END            & OR\\
\verb'\' & \verb';'   & EXTERN	       & OUT\\
\verb'_' & \verb':'   & FROM           & OVER\\
\verb'(' & \verb')'   & FUNCTION       & PROCEDURE\\
\verb'[' & \verb']'   & IF             & PROCESSORS\\
\verb'{' & \verb'}'   & IMPLEMENTATION & RESULT\\
\verb'/*' & \verb'*/' & IMPORT         & SHAPE\\
\end{tabular}}

\end{description}

\subsection*{Scope rules}

Every identifier occuring in a program must be introduced by a
declaration or import, unless it is a predeclared
identifier. Declarations also specify certain permanent properties of
an identifier, such as whether it is a constant, a type, a variable
(and its type), or a procedure.  The identifier is used to refer to
the associated entity.

The scope of an identifier is the block (module, procedure, function,
or statement) to which the declaration belongs and to which the
identifier is local. Scopes can be nested. The meaning of an
identifier is determined by the smallest enclosing scope in which the
identifier is declared. Entities referred to by declared identifiers
are called {\em visible}.

\subsection*{Name spaces}

Each scope $S$ consists of three {\em name spaces}:

\begin{enumerate}

\item The set $\TypeDefSet \subseteq \{(A, T) | A \in
\Language{\TC{Identifier}} \mbox{ and } T \in \Language{\NT{Type}}\}$
represents the set of user defined type definitions; i.e. $(A,T) \in
\TypeDefSet$ iff in S a type definition $A\ \T{=}\ T$ is visible. The
set $\TypeDefSet$ must define a function. An identifier $A$ such that
there exists a type $T$ and $(A, T) \in {\cal T}$ is called a {\em
declared} type.

\item The set of function and procedure identifiers ${\cal F}$ is
defined by the set of function and procedure declarations visible
within $S$.  An identifier in ${\cal F}$ is called a {\em declared}
procedure or function.

\item The set ${\cal D}$ of variable, view, and constant identifiers
defined by variable and constant declarations, or the variable
bindings of a selector visible within $S$. An identifier in ${\cal D}$
is called a {\em declared} variable, view or constant.

\end{enumerate}

\subsection*{Errors}

A {\em static error} is an error that a language implementation must
detect and report before program execution. Violations of the language
definition are static errors unless they are explicitely classified as
runtime errors.

A {\em checked runtime error} is an error that a language
implementation must detect and report. The method for reporting such
errors is implementation-dependent. For reasons of efficiency a
language implementation should provide a compiler option that
suppresses checking and reporting of runtime errors. The occurrence of
an {\em unchecked runtime error} may cause the subsequent behaviour of
a computation to be arbitrary.

