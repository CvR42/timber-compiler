% Written by:
% * Paul Dechering
% * Arnaud Poelman
% * Joachim Trescher
% * Hans de Vreught

We define the syntax of \Booster\ using an extended {\small BNF}. The
extensions have the following meaning:

\begin{tabular}{ll}
\\
{\em EBNF Construct} & {\em Meaning}\\
\\
\terminal{$\alpha$} & $\alpha$ is a terminal symbol\\
\optional{$\alpha$} & zero or one $\alpha$ \\
\compound{$\alpha$}  & $\alpha$ is a syntactical unit\\
$\alpha$\separation{$\beta$} & a non empty sequence of $\alpha$'s
separated by $\beta$'s \\
\any{$\alpha$} & an arbitrary number of $\alpha$'s\\
\iteration{$\alpha$} & one or more $\alpha$'s\\
$\alpha \beta$ & first $\alpha$ then $\beta$\\
$\alpha$\alternative $\beta$ & either $\alpha$ or $\beta$
\end{tabular}

\section{Modules}

\begin{production}
\NT{Booster}
   & \rewrite & \NT{DefinitionModule} \\
   & \alternative & \NT{ImplementationModule} \\
   & \alternative & \NT{AnnotationModule} \\
   & \point \\
\NT{DefinitionModule}
   & \rewrite & \NT{DefinitionHeader} \NT{DefinitionBody} \\
   & \point \\
\NT{DefinitionHeader}
   & \rewrite & \T{DEFINITION} \T{MODULE} \TC{Identifier} \T{;} \\
   & \point \\
\NT{DefinitionBody}
   & \rewrite & \any{\NT{DefinitionUnit}} \T{END} \TC{Identifier} \T{.} \\
   & \point \\
\NT{ImplementationModule}
   & \rewrite & \NT{ImplementationHeader} \NT{ImplementationBody} \\
   & \point \\
\NT{ImplementationHeader}
   & \rewrite & \optional{\T{IMPLEMENTATION}} \T{MODULE} \TC{Identifier} \T{;} \\
   & \point \\
\NT{ImplementationBody}
   & \rewrite & \any{\NT{ImplementationUnit}} \optional{\T{BEGIN} \NT{StatementList}} \\
   & & \tab \T{END} \TC{Identifier} \T{.} \\
   & \point \\
\NT{AnnotationModule} & \rewrite & \NT{AnnotationHeader} \NT{AnnotationBody} \\
   & \point \\
\NT{AnnotationHeader} & \rewrite & \T{ANNOTATION} \T{MODULE} \TC{Identifier} \T{;} \\
   & \point \\
\NT{AnnotationBody} & \rewrite & \any{\NT{AnnotationUnit}}
                      \optional{\T{BEGIN} \NT{AnnotationList}}\\
   & & \tab \T{END} \TC{Identifier} \T{.} \\
   & \point \\
\end{production}

\section{Units}

\begin{production}
\NT{DefinitionUnit}
   & \rewrite & \NT{ImportDefinition} \\
   & \alternative & \NT{ConstDefinition} \\
   & \alternative & \NT{TypeDefinition} \\
   & \alternative & \NT{VarDeclaration} \\
   & \alternative & \NT{ProcedureHeader} \\
   & \alternative & \NT{FunctionHeader} \\
   & \point \\
\NT{ImplementationUnit}
   & \rewrite & \NT{ImportDefinition} \\
   & \alternative & \NT{ConstDefinition} \\
   & \alternative & \NT{TypeDefinition} \\
   & \alternative & \NT{VarDeclaration} \\
   & \alternative & \NT{Procedure} \\
   & \alternative & \NT{Function} \\
   & \point \\
\NT{AnnotationUnit}
   & \rewrite & \NT{ImportDefinition}\\
   & \alternative & \NT{MachineDeclaration}\\
   & \point \\
\NT{ImportDefinition}
   & \rewrite & \optional{\T{FROM} \TC{Identifier}} \T{IMPORT}
     \compound{\TC{Identifier} \separation \T{,}} \T{;} \\ 
   & \point \\ 
\NT{ConstDefinition}
   & \rewrite &  \T{CONST} \NT{ConstList} \T{END} \T{;} \\
   & \point \\
\NT{TypeDefinition}
   & \rewrite &  \T{TYPE} \NT{TypeList} \T{END} \T{;} \\
   & \point \\
\NT{VarDeclaration}
   & \rewrite & \T{VAR} \NT{DeclarationList} \T{END} \T{;} \\
   & \point \\
\NT{MachineDeclaration}
   & \rewrite & \T{MACHINE} \NT{DeclarationList} \T{END} \T{;} \\
   & \point \\
\NT{Procedure}
   & \rewrite & \NT{ProcedureHeader} \NT{Body} \\
   & \point \\
\NT{ProcedureHeader}
   & \rewrite & \optional{\T{EXTERN}} \T{PROCEDURE} \TC{Identifier}\\
   &          & \tab  \T{(} \optional{\NT{ArgumentList}} \T{)} \T{;} \\  
   & \point \\
\NT{Function}
   & \rewrite & \NT{FunctionHeader} \NT{Body} \\
   & \point \\
\NT{FunctionHeader}
   & \rewrite & \optional{\T{EXTERN}} \T{FUNCTION} \TC{Identifier}
                  \T{(}\optional{\NT{ArgumentList}}\T{)} \\ 
   &  & \tab \T{RESULT} \NT{SingleDeclaration} \T{;} \\
   & \point \\
\NT{Body}
   & \rewrite & \optional{\NT{VarDeclaration}} \T{BEGIN}
   \NT{StatementList}\\
   &  & \tab \T{END} \TC{Identifier} \T{;} \\ 
   & \point \\
\end{production}

\section{Declarations}

\begin{production}
\NT{ConstList}
   & \rewrite & \optional{\TC{Identifier} \T{=} \NT{Expression}} \separation\ \T{;} \\
   & \point \\
\NT{TypeList}
   & \rewrite & \optional{\TC{Identifier} \T{=} \NT{Type}} \separation\ \T{;} \\
   & \point \\
\NT{DeclarationList}
   & \rewrite & \optional{\compound{\TC{Identifier} \separation\ \T{,}} \T{:} \NT{Type}} \separation\ \T{;} \\
   & \point \\
\NT{SingleDeclaration}
   & \rewrite & \TC{Identifier} \T{:} \NT{Type} \\
   & \point \\
\NT{ArgumentList}
   & \rewrite & \NT{Argument} \separation\ \T{;} \\
   & \point \\
\NT{Argument}
   & \rewrite & \NT{FlowType} \compound{\TC{Identifier} \separation\ \T{,}} \T{:} \NT{Type} \\
   & \point \\
\NT{FlowType}
   & \rewrite & \T{IN} \alternative\ \T{OUT} \alternative\ \T{INOUT} \\
   & \point \\
\NT{Type}
   & \rewrite & \TC{Identifier} \\
   & \alternative & \NT{TypeConstructor} \\
   & \point \\
\NT{TypeConstructor}
   & \rewrite & \NT{Kind} \verb*"{"* \NT{CardinalityList} \verb*"}" \T{OF}
                \NT{Type} \\ 
   & \point \\
\NT{Kind}
   & \rewrite & \T{SHAPE}\\
   & \alternative & \T{VIEW}\\
   & \point\\
\NT{CardinalityList}
   & \rewrite & \NT{Cardinality} \separation\ \T{\#} \\
   & \point \\
\NT{Cardinality}
   & \rewrite & \T{CONST} \TC{Identifier} \\
   & \alternative & \optional{\TC{Identifier} \T{:}} \NT{Expression} \\
   & \alternative & \T{*} \\
   & \point \\
\end{production}

\section{Flow of Control}

\begin{production}
\NT{AnnotationList}
   & \rewrite & \optional{\NT{AnnotationStatement}} \separation\ \T{;} \\
   & \point \\
\NT{StatementList}
   & \rewrite & \optional{\NT{Statement}} \separation\ \T{;} \\
   & \point \\
\NT{Statement}
   & \rewrite & \NT{ControlFlowStatement} \\
   & \alternative & \NT{ViewStatement} \\
   & \alternative & \NT{ContentStatement} \\
   & \alternative & \NT{ProcedureCall} \\
   & \point \\
\NT{ControlFlowStatement}
   & \rewrite & \NT{WhileStatement} \\
   & \alternative & \NT{IterStatement} \\
   & \alternative & \NT{IfStatement} \\
   & \point \\
\NT{WhileStatement}
   & \rewrite & \T{WHILE} \NT{Expression} \T{DO} \NT{StatementList} \T{END} \\
   & \point \\
\NT{IterStatement}
   & \rewrite & \T{ITER} \NT{Expression} \T{TIMES} \NT{StatementList} \T{END} \\
   & \alternative & \T{ITER} \TC{Identifier} \T{OVER} \NT{Expression} \\
   & & \tab \T{DO} \NT{StatementList} \T{END} \\
   & \point \\
\NT{IfStatement}
   & \rewrite & \T{IF} \NT{Expression} \T{THEN} \NT{StatementList} \optional{\NT{ElsePart}} \T{END} \\
   & \point \\
\NT{ElsePart}
   & \rewrite & \T{ELSE} \NT{StatementList} \\
   & \alternative & \T{ELSEIF} \NT{Expression} \T{THEN} \NT{StatementList} \optional{\NT{ElsePart}} \\
   & \point \\
\end{production}

\section{Statements}

\begin{production}
\NT{AnnotationStatement}
   & \rewrite & \NT{AnnotationDesignator} \T{$\mathtt{<-}$} \NT{Structure} \\
   & \point \\
\NT{AnnotationDesignator} 
   & \rewrite & \NT{QualIdent} \optional{\T{\{} \NT{CardinalityList} \T{\}}} \\
   & \point \\
\NT{ViewStatement}
   & \rewrite & \NT{ViewDesignator} \T{$<-$} \NT{Structure} \\
   & \point \\
\NT{ViewDesignator} 
   & \rewrite & \NT{QualIdent} \optional{\T{\{} \NT{CardinalityList} \T{\}}} \\
   & \point \\
\NT{ContentStatement}
   & \rewrite & \NT{ContentDesignator} \NT{Assignment} \NT{Expression} \\
   & \point \\
\NT{Assignment}
   & \rewrite & \T{:=} \\
   & \alternative & \T{||=} \\
   & \point \\
\NT{ContentDesignator}
   & \rewrite & \NT{Structure} \\
   & \point \\
\NT{SelectorList}
   & \rewrite & \NT{Selector} \separation\ \T{,} \\
   & \point \\
\NT{Selector}
   & \rewrite & \optional{\TC{Identifier} \T{:}} \NT{Expression} \\
   & \point \\
\NT{ProcedureCall}
   & \rewrite & \NT{QualIdent} \T{(} \optional{\NT{ExpressionList}} \T{)}  \\
   & \point \\
\end{production}

\section{Expressions}

\begin{production}
\NT{Expression}
   & \rewrite & \NT{Expression} \TC{RelOp} \NT{ArithmeticExpression} \\
   & \alternative & \NT{ArithmeticExpression} \\
   & \point \\
\NT{ArithmeticExpression}
   & \rewrite & \NT{ArithmeticExpression} \TC{AddOp} \NT{Term} \\
   & \alternative & \NT{Term} \\
   & \point \\
\NT{Term}
   & \rewrite & \NT{Term} \TC{MulOp} \NT{Factor} \\
   & \alternative & \NT{Factor} \\
   & \point \\
\NT{Factor}
   & \rewrite & \optional{\TC{UnOp}} \T{(} \compound{\NT{Expression} \separation \T{,}} \T{)}  \\
   & \alternative &  \optional{\TC{UnOp}} \T{\{} \NT{ExpressionList} \T{\}} \\
   & \alternative & \optional{\TC{UnOp}} \NT{Structure} \\
   & \alternative & \optional{\TC{UnOp}} \TC{Number} \\
   & \alternative & \optional{\TC{UnOp}} \T{\$} \\
   & \alternative & \TC{StringLiteral} \\
   & \alternative & \T{\_} \\
   & \point \\
\NT{ExpressionList}
   & \rewrite & \NT{Expression} \separation\ \T{,} \\
   & \point \\
\NT{Structure}
   & \rewrite & \NT{FunctionCall} \\
   & \alternative & \NT{Designator} \\
   & \point \\
\NT{FunctionCall}
   & \rewrite & \optional{\TC{Identifier} \T{.}} \TC{Identifier} \T{(}
   \optional{\NT{ExpressionList}} \T{)}\\
   & & \tab  \optional{\T{[} \NT{SelectorList} \T{]}} \\ 
   & \point \\
\NT{Designator} 
   & \rewrite & \optional{\TC{Identifier} \T{.}} \TC{Identifier} \optional{\T{[} \NT{SelectorList \T{]}}} \\
   & \point \\
\end{production}

\section{Tokens}

\begin{production}
\TC{RelOp}
   & \rewrite & \T{>} \alternative\ \T{>=} \alternative\ \T{<>} \alternative\ \T{=} \alternative\ \T{<=} \alternative\ \T{<} \alternative\ \T{|} \\
   & \point \\
\TC{AddOp}
   & \rewrite & \T{OR} \alternative\ \T{+} \alternative\ \T{-} \alternative\ \T{\&} \\
   & \point \\
\TC{MulOp}
   & \rewrite & \T{AND} \alternative\ \T{DIV} \alternative\ \T{MOD} \alternative\ \T{*} \alternative\ \T{/} \alternative\ \T{..} \\
   & \point \\
\TC{UnOp}
   & \rewrite & \T{NOT} \alternative\ \T{+} \alternative\ \T{-} \alternative\ \T{$\backslash$} \\
   & \point \\
\TC{Identifier}
   & \rewrite & \TC{Letter} \any{\compound{\TC{Letter} \alternative\ \TC{Digit}}} \\
   & \point \\
\TC{Letter}
   & \rewrite & \T{A} \class\ \T{Z} \alternative\ \T{a} \class\ \T{z} \\
   & \point \\
\TC{Number}
   & \rewrite & \iteration{\TC{Digit}} \optional{\T{.} \iteration{\TC{Digit}}} \optional{\T{E} \optional{\T{+} \alternative\ \T{-}} \iteration{\TC{Digit}}} \\
   & \point \\
\TC{Digit}
   & \rewrite & \T{0} \class\ \T{9} \\
   & \point \\
\TC{StringLiteral}
   & \rewrite & \T{'} \any{\compound{\TC{Character} \alternative\ \TC{EscapedCharacter}}} \T{'} \\
   & \point \\
\TC{EscapedCharacter}
   & \rewrite & \T{$\backslash$b} \alternative\ \T{$\backslash$f} \alternative\ \T{$\backslash$n} \alternative\ \T{$\backslash$r} \alternative\ \T{$\backslash$t} \alternative\ \T{$\backslash$'} \alternative\ \T{$\backslash\backslash$} \\
   & \alternative & \T{$\backslash$} \TC{Digit} \optional{\TC{Digit}
   \optional{\TC{Digit}}} \\
   & \point \\
\TC{Character}
   & \rewrite & $\Sigma\setminus$\compound{\T{$\backslash$} \alternative\ \T{'}} \\
   & \point \\
\end{production}

\noindent
$\Sigma$ contains all printable ASCII characters \T{ } \class\ \T{\~{}}.

\section{White Space}

White space outside string literals may be used to separate tokens and is no
part of the language. It should be filtered out before parsing starts under
the condition that separated tokens remain separated. Beside space, tab,
linefeed, carriage return, and formfeed comments also count as white space.
There are two types of comments:
\begin{itemize}
\item A comment starts with \T{//} and ends at the end of the line. This is a
	single line comment.
\item A comment starts with \T{/*} and ends with \T{*/}. These comments may
	be nested.
\end{itemize}
