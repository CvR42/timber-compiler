% Written by:
% * Joachim Trescher

\section{Expressions}

An expression prescribes a computation that produces a value,
variable, or index set. Syntactically, an expression is either an
operand, or an operation applied to arguments, which are themselves
expressions. An expression is evaluated by recursively evaluating its
arguments and performing the operation. The order of argument
evaluation is undefined.

\subsection*{Compatibilities}

An expression $e$ of type $T_e$ is {\em type compatible} with a
designator $d$ of type $T_d$ if $T_d$ is compatible with $T_e$, or
$T_e$ is compatible with $T_d$. A list of expressions $\langle
e_1,\ldots,e_n\rangle$ with types $T_{e_i}, 1 \leq i \leq n,$ {\em
matches} a list of designators $\langle d_1,\ldots,d_m\rangle$ with
types $T_{d_i}, 1 \leq i \leq m,$ iff $n = m$, and $T_{e_i}$ is type
compatible with $T_{d_i}, 1 \leq i \leq m.$ An expression $e$ of type
$T_e$ is {\em assignment compatible} with a designator $d$ of type
$T_d$ iff $T_d$ includes $T_e$.

The type and assignment compatibility relations are relations on the
syntactical representation of types and expressions, respectively.
These relations can be computed and checked statically. However, at
run-time all variables in the representation of the {\em actual type}
of a designator or an expression are definite.

We say the type of an expression $e$ with the static type $T_e$ and
the actual type $AT_e$ is {\em compatible} with the type of a
designator $d$ with the static type $T_d$ and the actual type $AT_d$,
iff $T_d$ is compatible to $T_e$ and $AT_d$ is type compatible with
$AT_e$. We say the type of an expression $e$ with the static type
$T_e$ and the actual type $AT_e$ is {\em assignable} with the type of
a designator $d$ with the static type $T_d$ and the actual type
$AT_d$, iff $T_d$ includes $T_e$ and $AT_d$ includes $AT_e$.

If it is required that an expression $e$ is compatible with a
designator $d$, then it is a static error if $T_e$ is not compatible to
$T_d$, and it is a checked run-time error if $AT_e$ is not compatible
with $AT_d$. If it is required that an expression $e$ is assignable to
a designator $d$, then it is a static error if $T_d$ does not include
$T_e$, and it is a checked run-time error if $AT_d$ does not include
$AT_e$. If it is required that a list of expressions matches a list of
designators then it is a static error if the type of one of the
designators is not type compatible to the type of the corresponding
expression, and it is a checked run-time error if the actual type of
one of the designators is not compatible with the actual type of the
corresponding expression.

\subsection*{Operators}

The operators that have special syntax are classified and listed in
order of decreasing binding power in the following table:

\begin{tabular}{llll}
\hlf
\T{x.a} & infix qualification\\
\T{f(x)}
  & function application\\
\T{a[i]}
  & subscripting and selection\\
\T{+} \T{-}
  & prefix arithmetical operators\\
\T{*} \T{/} \T{DIV} \T{MOD}
  & infix arithmetical operators\\
\T{+} \T{-}
  & infix arithmetical operators\\
\T{=} \T{<>} \T{<} \T{>} \T{<=} \T{>=} 
  & infix relational operators\\
\T{NOT}
  & prefix boolean operator\\
\T{AND}
  & infix boolean operator\\
\T{OR}
  & infix boolean operator\\
\hlf
\end{tabular}

\noindent All infix operators are left associative. Parentheses can be
used to override the precedence rules. Operators without special
syntax are functional. An application of a functional operator has the
form {\sf op(args)}, where {\sf op} is the operation and {\sf args} is
a list of argument expressions that match the formal parameter list of
the functional operator {\sf op}. An application of a functional
operator is an expression. The type of the expression is the result
type of the functional operator {\sf op}.

\subsection*{Operands}

An operand is either a designator, function designator, selection,
constant, or a literal. Literals are string literals, boolean
literals, numeric literals, or shape literals.

A string literal is an element of \Language{\TC{StringLiteral}}. The
type of a string literal is \T{STRING}.

A boolean literal is one of the predefined identifiers \texttt{TRUE}
or \texttt{FALSE}.

Numeric literals denote non-negative natural, real, or complex
values. The type of a numeric literal is the smallest numeric type,
that contains the denoted value. A literal of type \T{NATURAL} is a
non-empty sequence of the decimal digits {\sf 0} through {\sf 9}. The
digits are interpreted in base 10. The value of the literal must be at
most \texttt{MAXNATURAL}.

A literal of type \T{REAL} has the form {\sf decimal E exponent},
where {\sf decimal} is a non-empty sequence of decimal digits followed
by a decimal point followed by a non-empty sequence of decimal digits,
and {\sf exponent} is a non-empty sequence of decimal digits
optionally beginning with a \T{+} or \T{-}. The literal denotes
decimal times $10^{\sf exponent}$. If {\sf E exponent} is omitted,
{\sf exponent} defaults to 0. As a shorthand notation it is acceptable
to use a natural literal to represent {\sf decimal} of a natural
valued literal of type \T{REAL}.

A literal of type \T{COMPLEX} is a list of two expressions enclosed in
parentheses \T{(} and \T{)}. The first expression specifies the real
part of a complex number and the second specifies the imaginary part
of a complex number.

A shape literal is a list of expressions enclosed in curly braces
\T{\{} and \T{\}}. The expressions either evaluate all to numeric
values, to boolean values, to string values, or to shape values. 

If the expressions evaluate to numeric values then the base type of
the shape literal is the smallest numeric type that includes all the
values of the sequence, and the form of the shape literal is the
sequence $\langle l \rangle$, where $l$ is the length of the
expression list denoted by the shape literal.

If the expressions evaluate to string values then the base type of the
shape literal is \T{STRING}, and the form of the shape literal is the
sequence $\langle l \rangle$, where $l$ is the length of the
expression list denoted by the shape literal.

If the expressions evaluate to boolean values then the base type of
the shape literal is \T{BOOLEAN}, and the form of the shape literal is
the sequence $\langle l \rangle$, where $l$ is the length of the
expression list denoted by the shape literal.

If the expressions evaluate to shape values, then we require that all
shape values have the same form $F$, and the base type of all shape
values is either \T{STRING}, \T{BOOLEAN}, or a numeric type. The base
type of the shape literal is \T{STRING}, or \T{BOOLEAN} in the first
two cases, and the smallest numeric type that includes the base type
of all shape values in the latter case. The form of the shape literal
is the sequence $\langle l \rangle \concat F$, where $l$ is the length
of the expression list denoted by the shape literal.


\subsection*{Set expressions}

A set expression prescribes a computation that produces a selection of
indices.  Syntactically, a set expression is an {\em index selection}
or a set operator applied to arguments which are themselves set
expressions. A set expression is evaluated by recursively evaluating
its arguments and applying the set operator. The order of argument
evaluation is undefined.

The value of a set expression is not a member of any \Booster\ type.
Therefore, set expressions are meaningful exclusively in the context
of a selector list of a subscript. If a set expression $S$ occurs on
the {\em n-th} position of a selector list of a subscript, and $c_n$
is the {\em n-th} element of the form of the subscripted variable,
then $C_S \ \widehat{=} \ \{ 0,\ldots c_{n-1}\}$ is called the {\em
context} of $S$, and $L_{C_S} \ \widehat{=} \ n$ its length.

The selectors and set operators are listed in order of decreasing
binding power in the following table:

\begin{tabular}{ll}
\hlf
\verb'..' & infix range selection\\
\verb'_' & don't care selection\\
\verb'$' & maximum selector\\
\hlf
\verb'\' & prefix difference operator\\
\verb'|' & infix union operator\\
\verb'&' & infix intersection\\
\end{tabular}

\noindent The range selection takes two arguments of type \T{NATURAL}
and specifies the index selection $\{a,\ldots,b\}$ where $a$ is the
value of the first argument and $b$ is the value of the second
argument. If $a > b$ the empty selection is specified. It is required
that $b < L_{C_S}$. The maximum selector is an expression of type
\T{NATURAL} that evaluates to $L_{C_\$}-1$. The don't care selection
is treated as a shorthand for the set expression \T{0..\$}.

The difference operator takes a set expression as an argument and
specifies the selection $D \quad \widehat{=} \quad C \setminus S$,
where $C$ is the context of the set expression and $S$ the selection
specified by its argument. The union and intersection operators take
two set expressions as arguments and specify the selection obtained by
the union or intersection of the argument values.

\subsection*{Designators}

An identifier is a {\em writable} designator if it is declared as a
variable, or a formal parameter.  An identifier is a {\em readonly}
designator if it is a generic constant, an index variable, or a
declared constant.

The only expressions that produce designators are qualification,
function application, subscripting, and selection. The following table
defines these operations and specifies the conditions under which they
produce designators.

\begin{description}

\item[{\sf a[e$_{\sf 1}$,\ldots,e$_{\sf m}$]}] 

If all {\sf e$_{\sf i}$} are expressions of type {\tt NATURAL}, $1
\leq i \leq m$, and the writable designator {\sf a} has a shape type
with form $\langle n_1,\ldots,n_m \rangle \concat F$, then {\sf
a[e$_{\sf 1}$,\ldots,e$_{\sf m}$]} is a writable designator that
denotes the element of {\sf a} with index $[i_1,\ldots,i_m]$, where
$i_j$ is the value of expression {\sf e$_{\sf j}$}. It is required
that $0 \leq i_j \leq n_j-1$ for $ 1 \leq j \leq m$. The type of {\sf
a[e$_{\sf 1}$,\ldots,e$_{\sf m}$]} is the shape type with form $F$ and
the same base type as {\sf a}.

If all {\sf e$_{\sf i}$} are set expressions, $1 \leq i \leq m$, and
the writable designator {\sf a} has a shape type with form $\langle
n_1,\ldots,n_m \rangle \concat F$, then {\sf a[e$_{\sf
1}$,\ldots,e$_{\sf n}$]} is a writable designator that denotes the
shape that contains the elements of {\sf a} whose index are elements
of the set $\{[i_1,\ldots,i_m] \mid i_j \in E_j\}$, where $E_j$ is the
value of the set expression {\sf e$_{\sf j}$}. The type of {\sf
a[e$_{\sf 1}$,\ldots,e$_{\sf n}$]} is the shape type with form
$\langle l_1,\ldots,l_m\rangle \concat F$, where $l_i$ is the
cardinality of the set $E_i$, $1 \leq i \leq m$, and the same base
type as {\sf a}. It is required that $\forall j \in \{1,\ldots,m\}
\bullet \forall i \in E_j \bullet i \in \{0,\ldots,n_j-1\}$.

Note, an expression with the syntactical form \T{a[\ldots][\ldots]}
is not an element of the \Booster\ language.


\item[{\sf M.x}]

If {\sf M} is the name of an imported module, then {\sf M.x} denotes
the entity named {\sf x} in the module {\sf M}. In this case {\sf M.x}
is a designator if {\sf x} is declared as a variable; such a
designator is always writable.


\item[{\sf f(a$_{\sf 1}$,\ldots,a$_{\sf n}$)}]

If {\sf f} is declared as a function that returns a result of a view
type, and the expressions {\sf a$_{\sf 1}$,\ldots,a$_{\sf n}$} match the
formal parameters of the function {\sf f} then {\sf
f(a$_{\sf 1}$,\ldots,a$_{\sf n}$)} denotes a writable designator.

\item[{\sf f(a$_{\sf 1}$,\ldots,a$_{\sf n}$)}]

If {\sf f} is declared as a function and the expressions {\sf a$_{\sf
1}$,\ldots,a$_{\sf n}$} match the formal parameters of the function
{\sf f} then the function designator {\sf f(a$_{\sf 1}$,\ldots,a$_{\sf
n}$)} as a constituent of an expression denotes an operand of the
result type of {\sf f}.

\end{description}

\subsection*{Expression compatibility}

Some operators are applicable to operands of various types, denoting
different operations. In these cases, the actual operation is
identified by the type of the operands. For a given operator, the
types of its operands are {\em expression compatible} if they conform
to tables given below, which also show the result types of the
corresponding expressions. 

These tables use the auxiliary functions \coerce\ and \MaxDim. The
function \coerce\ chooses from a set of intrinsic numeric types an
element that includes all other elements. The function \MaxDim\
computes from two forms a form that includes both and contains as much
information as possible (i.e.\ replaces where possible variables by
constant expressions). More formally:

\[
\begin{array}{rcl}
\multicolumn{3}{c}{
\coerce: 2^{\Language{\NT{Types}}} \rightarrow \Language{\NT{Types}}}\\
\hlf
\coerce(\mbox{\em TypeSet\/}) & \widehat{=} & \left\{
  \begin{array}{ll}
  T & \mbox{such that } T \in \mbox{\em TypeSet\/} \wedge \\
    &  \forall T' \in \mbox{\em TypeSet} \bullet T
      \sqsupseteq T' \\
  \end{array}\right.\\
\end{array}
\]

\[
\begin{array}{rcll}
\multicolumn{3}{c}{
\MaxDim: \Language{\NT{Types}} \times \Language{\NT{Types}}
\rightarrow \Forms}\\
\hlf
\MaxDim(\langle d_1,\ldots,d_m\rangle, \langle e_1,\ldots,e_n\rangle) 
    & \widehat{=} & \langle f_1,\ldots,f_k\rangle
\end{array}
\]
where $k = \mbox{\em max}(\{n,m\})$, and for $1 \leq i \leq k$ holds
$f_i = d_i$ if $d_i$ is a natural value or $n < i$, else $f_i = e_i$
if $e_i$ is a natural value or $m < i$, $f_i = X$ otherwise, where $X$
is a fresh free variable from \freevar. For example, consider the two
forms $F_1 = \langle 3, X, 5\rangle$ and $F_2 = \langle Y, 4 \rangle$
then we have $\MaxDim(F_1, F_2) = \langle 3,4,5\rangle$.

The following table specifies the result types of an application of an
unary operator:

\begin{tabular}{llll}
\\
{\em operator} & {\em operand type} & {\em result type}\\
\hlf
\T{+} & $T_1$ & $T_1$\\
{\em condition} & \T{COMPLEX} $\sqsupseteq \bt(T_1)$\\ 
\hlf
\T{-} & $T_1$ & $T$ such that\\
{\em condition} & \T{COMPLEX} $\sqsupseteq \bt(T_1)$
& \form$(T)$ = \form$(T_1)$\\
& & \bt($T$) = \coerce(\{\bt($T_1$), \T{INTEGER}\})\\
\hlf
\T{NOT} & $T_1$ & $T_1$\\
{\em condition} & \bt$(T_1)$ = \T{BOOLEAN}\\
\\
\end{tabular}

For the operand types $T_1$ and $T_2$ of the binary arithmetical and
boolean operators it is required that either $T_1$ includes $T_2$ or
$T_2$ includes $T_1$.

\begin{tabular}{llll}
\\
{\em operator} & {\em 1.\ operand} & {\em 2.\ operand} 
& {\em result type}\\
\hlf
\T{+} \T{*} & $T_1$ & $T_2$ & $T$ such that\\
{\em conditions} & \multicolumn{2}{l}{\T{COMPLEX} $\sqsupseteq$
\bt($T_1$)}
& 1.\ $\form(T) = \mbox{\em MaxDim}(T_1, T_2)$\\
& \multicolumn{2}{l}{\T{COMPLEX} $\sqsupseteq$ \bt($T_2$)}
& 2.\ \bt($T$) = $\coerce(\{\bt(T_1),\bt(T_2)\})$\\
\hlf
\T{-} & $T_1$ & $T_2$ & $T$ such that\\
{\em conditions} & \multicolumn{2}{l}{\T{COMPLEX} $\sqsupseteq$
\bt($T_1$)}
& 1.\ $\form(T) = \mbox{\em MaxDim}(T_1, T_2)$\\
& \multicolumn{2}{l}{ \T{COMPLEX} $\sqsupseteq$ \bt($T_2$)}
& 2.\ \bt($T$) =$\coerce(\{\bt(T_1),\bt(T_2),$\\
& & & \quad $\T{INTEGER}\})$\\
\hlf
\T{/} & $T_1$ & $T_2$ & $T$ such that\\
{\em conditions} & \multicolumn{2}{l}{\T{COMPLEX} $\sqsupseteq$
\bt($T_1$)}
& 1.\ $\form(T) = \mbox{\em MaxDim}(T_1, T_2)$\\
& \multicolumn{2}{l}{ \T{COMPLEX} $\sqsupseteq$ \bt($T_2$)}
& 2.\ \bt($T$) = $\coerce(\{\bt(T_1),\bt(T_2),$\\
& & & \quad $\T{REAL}\})$\\
\hlf
\T{MOD} & $T_1$ & $T_2$ & $T$ such that\\
{\em conditions} & \multicolumn{2}{l}{\T{INTEGER} $\sqsupseteq$
\bt($T_1$)}
& 1.\ $\mbox\form(T) = \mbox{\em MaxDim}(T_1, T_2)$\\
& \multicolumn{2}{l}{ \T{INTEGER} $\sqsupseteq$ \bt($T_2$)}
& 2.\ \bt($T$) = \T{NATURAL}\\
\hlf
\T{DIV} & $T_1$ & $T_2$ & $T$ such that\\
{\em conditions} & \multicolumn{2}{l}{\T{INTEGER} $\sqsupseteq$
\bt($T_1$)}
& 1.\ $\mbox\form(T) = \mbox{\em MaxDim}(T_1, T_2)$\\
& \multicolumn{2}{l}{ \T{INTEGER} $\sqsupseteq$ \bt($T_2$)}
& 2.\ \bt($T$) = $\coerce(\{\bt(T_1), \bt(T_2)\})$\\
\hlf
\T{OR} \T{AND} & 
$T_1$ & $T_2$ & $T$ such that\\
{\em conditions} & \multicolumn{2}{l}{
\bt($T_1$) = \T{BOOLEAN}}
& 1.\ $\mbox\form(T) = \mbox{\em MaxDim}(T_1, T_2)$\\
& \multicolumn{2}{l}{\bt($T_2$) = \T{BOOLEAN}}
& 2.\ \bt($T$) = \T{BOOLEAN}\\
\hlf
\end{tabular}

The binary relational operators accept only operands whose type is
included by \T{REAL} except the tests for (in)equality which may be
applied to compatible multi-dimensional operands. To enable the use of
these expressions in the condition of a conditional or repetitive
statement the result type of such an expression is a scalar
\T{BOOLEAN}.

\begin{tabular}{llll}
\\
{\em binary operator} & {\em 1.\ operand} & {\em 2.\ operand} 
& {\em result type}\\
\hlf
\T{>} \T{>=}\\
\T{<} \T{<=} & $T_1$ & $T_2$ & \T{BOOLEAN}\\
{\em conditions} & \multicolumn{2}{l}{
$T_1 \sqsupseteq T_2 \vee T_2 \sqsupseteq T_1$}\\
\hlf
\T{<>} \T{=} & $T_1$ & $T_2$ & \T{BOOLEAN}\\
{\em conditions} & \multicolumn{2}{l}{
$\bt(T_1) \sqsupseteq \bt(T_2)$}\\
& \multicolumn{2}{l}{ 
  $T_1 \approx T_2 \vee T_2 \approx T_1$}\\
\\
\end{tabular}

It is a static error if in an expression the operands are not
expression compatible. It is a checked run-time error if the actual
types of the operands of an expression are not expression compatible.
