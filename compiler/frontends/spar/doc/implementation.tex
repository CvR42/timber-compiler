\documentclass{article}
\newcommand{\spar}{{\sf Spar}}
\title{Spar implementation issues}
\author{C. van Reeuwijk}
\begin{document}
\maketitle
\section{Introduction}
This document discusses the more complicated issues of the implementation
of {\spar}.
\section{Compiler phases}
The {\spar} frontend consists of the following phases:
\begin{itemize}
\item Parsing. Implemented in \verb'parser.y'. Parses an input file,
     and constructs an abstract syntax tree.
\item Lowering. Implemented in \verb'lower.ct'. Ensures the following
      Spar language constructs are rewritten: \verb'for()' and
      \verb'foreach()' statements in classical notation; expressions
      with side-effects, and declarations after statements.
\item Name mangling and symbol table construction. Implemented in
     \verb'mangle.ct'. Rewrites the
      program code, so that a single program-wide symbol table can be used.
      When necessary, symbols are renamed so to avoid name clashes.
      Also constructs the global symbol table.
\item Semantic checking. Implemented in \verb'check.ct'. The context-dependent
      language restrictions are enforced, and some warnings are generated.
\item Rewriting. Implemented in \verb'rewrite.ct'. Rewrites a couple
      of constructs to a form the backend can handle: currently only
      strict type casting, and TypeOf elimination.
\item Semantic checking. Implemented in \verb'check.ct'. The check is
      repeated as a check on the rewrite phase above.
\end{itemize}

\section{Module handling and name mangling}


\section{Evaluation order of expressions}
\subsection{Description of the problem}
Java prescribes the evaluation order of an expression. This evaluation
order is only of interest for expressions with side-effects (where an
exception is a side-effect).
For example:
\begin{verbatim}
int i = 2;
int j = (i=3)*i;
...
\end{verbatim}
will cause \verb'j' to be 9, since binary expressions are evaluated
from left to right. In C and C++, \verb'j' may also have the value 6,
since the evaluation order is explicitly left unspecified.

The evaluation order of Vnus is also unspecified, but this is only of
interest for expressions containing function calls, since these are the
only expressions that may have side effects. Other side effects in
Spar expressions must have been eliminated before they reach Vnus,
see below.

\subsection{Solution}
To ensure that the evaluation order of Java is observed, statements
like the one above are rewritten to a form where the evaluation order
is ensured. This is done by isolating critical expressions in separate
assignment statements, that are evaluated in a known order.

If it is known that an expression has no side effects, or is evaluated
in a fixed order, it is not modified. This is done for cosmetic reasons:
this way the Vnus code is much clearer, and is easier to inspect.

The following expressions are known to have no side effects (are 'pure'):
\begin{itemize}
\item Literals.
\item Variable references.
\item Binary operators on pure sub-expressions.
\item Parameter list constructors with pure sub-expressions.
\item Field references with a pure class expression.
\item etc.
\end{itemize}

For example, the previous example must be rewritten to something like:
\begin{verbatim}
int i = 2;
i = 3;
{
    int j = i*i;
    ...
}
\end{verbatim}
while the statements
\begin{verbatim}
int i = 2;
int j = i*(i=3);
\end{verbatim}
must be rewritten to:
\begin{verbatim}
int i = 2;
int tmp = i;
int i = 3;
int j = tmp*i;
\end{verbatim}
\section{Side-effects in expressions}
\subsection{Description of the problem}
Evaluating Spar expressions may have side effects. For example, the following
statement is valid:
\begin{verbatim}
n = (a++)+3;
\end{verbatim}
Since Vnus only allows side effects in function evaluations, and
otherwise does not allow side effects in expressions, this statement
must be rewritten to:
\begin{verbatim}
{
    int tmp = a;

    a++;
    n = tmp+3;
}
\end{verbatim}
Where \verb'tmp' is a new variable name.
In initializations a slightly different translation must be used:
\begin{verbatim}
{
    int n = (a++)+3;
}
\end{verbatim}
must be rewritten
\begin{verbatim}
{
    int tmp1 = a;

    a++;
    {
        int n = tmp1+3;
    }
}
\end{verbatim}
\subsection{Solution specification}
The following expressions cause side effects:
\begin{itemize}
\item The assignment operator \verb'='.
\item Function calls (although we could exclude 'pure' functions).
\item The compound-assignment operators such as \verb'+='.
\item The increment and decrement operators.
\end{itemize}
In general, the following rewrites must be implemented to eliminate
side effects:
\begin{enumerate}
\item
To enforce strict Java evaluation order: any binary operator
expression that contains a side effect at its lefthand side, must be
exploded. Similarly for parameter lists to functions.
\item Rewrite pre-increment and pre-decrement expressions to compound-assignment
      form: \verb'++i' is rewritten to \verb'i += 1'
\item Rewrite post-increment and post-decrement expressions to use a
      temporary variable. For example, an expression containing a
      post-increment:
\begin{verbatim}
n = a++ +3;
\end{verbatim}
is rewritten to:
\begin{verbatim}
int tmp = a;
a += 1;
n = tmp +3;
\end{verbatim}
\item Rewrite (compound) assignment expressions as separate
      assignments and a use of the variable. For example:
\begin{verbatim}
int i = 2;
int j = (i=3)*i;
...
\end{verbatim}
is rewritten to:
\begin{verbatim}
int i = 2;
i = 3;
{
    int j = i*i;
    ...
}
\end{verbatim}
\end{enumerate}
\end{document}
