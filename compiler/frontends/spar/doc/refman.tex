\documentclass[a4paper]{article}
\newcommand{\langnm}{Sugar Vnus}
\newlength{\keywordwidth} % Used to format the list of keywords.
% take a more relaxed attitude to figures.
\renewcommand{\topfraction}{0.9}
\renewcommand{\bottomfraction}{0.9}
\renewcommand{\dbltopfraction}{0.9}
\renewcommand{\textfraction}{0.1}
\renewcommand{\floatpagefraction}{0.9}
\newcommand{\defn}[1]{{\em #1}\index{#1}}
\newcommand{\fnname}[1]{{\it #1}}
\newcommand{\mymod}{{\ \bmod \ }}
\newcommand{\bdiv}{\fnname{div}}
\newcommand{\floor}[1]{{\left \lfloor #1 \right \rfloor}}
\newcommand{\ceil}[1]{{\left \lceil #1 \right \rceil}}
\newcommand{\set}[1]{{\{\ #1\ \}}}
\newcommand{\nonterm}[1]{{\tt #1}}
\newtheorem{remark}{Remark}
\title{{\langnm} reference manual}
\author{C. van Reeuwijk}
\begin{document}
\maketitle
\section{Introduction}
This reference manual describes {\langnm}. The language has been developed to
provide a readable version of Vnus, but it may evolve into a separate
programming language.
\par
\begin{remark}
Since {\langnm} is still under development, a number of details have
not yet been finalized, or are still subject of debate. This is
explained in remarks like this.
\end{remark}
\par
This manual describes the language in full detail. It is not intended
as a tutorial.

\section{Lexical conventions}
\subsection{Tokens}
There are five classes of tokens: identifiers, keywords, constants,
operators, and separators.
Blanks, tabs, newlines, and comments (collectively,
"white space") are ignored except that they separate tokens.
Some white space is required to separate otherwise adjacent identifiers,
keywords, and constants.
If the input stream has been separated into tokens up to a given character,
the next token is the longest string of characters that can constitute
a token.
\subsection{Comments}
The characters \verb"//" introduce a comment, which terminates at the end
of the line.
The characters \verb"/*" introduce a comment, which terminates at the
characters \verb"*/".
Comment does not occur within strings.
\subsection{Identifiers}
An identifier is a sequence of letters and digits. The first character
must be a letter; the underscore \verb"_" counts as a letter.
All identifiers containing a sequence of two underscores are reserved for
use within the compiler, and should not be used in a program.
Upper and lower case letters are different. Identifiers may have any
length.
\subsection{Keywords}
The following identifiers are reserved for use as keywords, and may not be
used otherwise: 
\settowidth{\keywordwidth}{\tt redistribute1}
\begin{quote}
\makebox[\keywordwidth][l]{\tt and}
\makebox[\keywordwidth][l]{\tt barrier}
\makebox[\keywordwidth][l]{\tt blocksize}
\makebox[\keywordwidth][l]{\tt block}
\makebox[\keywordwidth][l]{\tt boolean}
\makebox[\keywordwidth][l]{\tt broadcast}
\makebox[\keywordwidth][l]{\tt complex}
\makebox[\keywordwidth][l]{\tt div}
\makebox[\keywordwidth][l]{\tt else}
\makebox[\keywordwidth][l]{\tt false}
\makebox[\keywordwidth][l]{\tt forall}
\makebox[\keywordwidth][l]{\tt forkall}
\makebox[\keywordwidth][l]{\tt for}
\makebox[\keywordwidth][l]{\tt function}
\makebox[\keywordwidth][l]{\tt getsize}
\makebox[\keywordwidth][l]{\tt if}
\makebox[\keywordwidth][l]{\tt inout}
\makebox[\keywordwidth][l]{\tt integer}
\makebox[\keywordwidth][l]{\tt in}
\makebox[\keywordwidth][l]{\tt isowner}
\makebox[\keywordwidth][l]{\tt mod}
\makebox[\keywordwidth][l]{\tt natural}
\makebox[\keywordwidth][l]{\tt not}
\makebox[\keywordwidth][l]{\tt of}
\makebox[\keywordwidth][l]{\tt or}
\makebox[\keywordwidth][l]{\tt out}
\makebox[\keywordwidth][l]{\tt owner}
\makebox[\keywordwidth][l]{\tt procedure}
\makebox[\keywordwidth][l]{\tt real}
\makebox[\keywordwidth][l]{\tt receive}
\makebox[\keywordwidth][l]{\tt returns}
\makebox[\keywordwidth][l]{\tt sender}
\makebox[\keywordwidth][l]{\tt send}
\makebox[\keywordwidth][l]{\tt shape}
\makebox[\keywordwidth][l]{\tt string}
\makebox[\keywordwidth][l]{\tt synchronize}
\makebox[\keywordwidth][l]{\tt true}
\makebox[\keywordwidth][l]{\tt view}
\makebox[\keywordwidth][l]{\tt want}
\makebox[\keywordwidth][l]{\tt while}
\end{quote}
In the recognition of keywords, case is ignored, so
\verb"Procedure", \verb"PROCEDURE" and \verb"PrOcEdUrE" are also
recognized as keywords.
\subsection{Constants}
There are several kinds of constants: integer, real, string and boolean.
Each has a data type; section \ref{s.types} discusses these.
\begin{remark}
For the moment little distinction is made between natural,
integer, and real constants. If a real constant is recognizable
as such (contains a decimal point or a "e" or "E"), this is noted,
otherwise it is assumed to be an integer constant. Partially for
this reason, the compiler
takes a relaxed attitude to mixing of differently typed expressions.
\end{remark}
\subsubsection{Integer constants}
An integer constant consists of a sequence of digits.
Signs (\verb'+' or \verb'-') in front
of these constants are not part of the constant, but are unary operators.
\par
Examples: \verb"12", \verb"42".
\subsubsection{Real constants}
A real constant consists of a sequence of digits, that is optionally followed
by a decimal point \verb"." and an other sequence of digits. This may
be followed by a exponent separator \verb"e" or \verb"E" followed by an
optionally signed sequence of digits.
Signs (\verb'+' or \verb'-') in front
of these constants are not part of the constant, but are unary operators.
\par
Examples: \verb"1", \verb"1.2", \verb"1e22", \verb"1.0e-33".
\subsubsection{String constants}
A string constant is a sequence of characters surrounded by double quotes.
Examples: \verb'""', \verb'"Hello world"'.
\subsubsection{Boolean constants}
There are two boolean constants, represented by the keywords {\tt true}
and {\tt false}. Note that, just as for other keywords, the case
is ignored.
\par
Examples: {\tt False}, {\tt TRUE}.
\subsection{Operators}
The following operators are recognized:
\begin{quote}
\verb"&"
\verb"*"
\verb"+"
\verb"-"
\verb".."
\verb"/"
\verb":"
\verb"::"
\verb":="
\verb"<"
\verb"<-"
\verb"<<"
\verb"<="
\verb"<>"
\verb"="
\verb">"
\verb">="
\verb">>"
\verb"\"
\verb"|"
\end{quote}
\subsection{Separators}
\begin{verbatim}
commas: "," | commas ","
\end{verbatim}
\subsection{Distributions}
\label{s.distributions}
\begin{verbatim}
distribution:
    "block"
|
    "block" expression
|
    "cyclic"
|
    "*"
|
   "+"
|
   "-"
;

distributions:
    <empty>
|
    "[" distribution-list opt_commas "]"
;
\end{verbatim}
Unlike other data-parallel languages, {\langnm} does
not allow explicit declaration of processor arrays. Instead, the
structure of the processor array is given as a command-line option
to the backend as a list of array sizes. From this specification two
implicit processor arrays are derived: a $m$-dimensional one, where
$m$ is the rank of the processor array, and a one-dimensional array
with the same number of processors.
For example, the backend can be invoked with

\begin{verbatim}
    vnus -t5,4,12
\end{verbatim}
This specifies a 3-dimensional processor array of $5 \times 4 \times 12$ elements,
and a 1-dimensional processor array of $5 \cdot 4 \cdot 12 = 240$ processors.
\par
The sizes of the processor array are available within the program
through the predefined shape 'sizesProcessorArray'. Thus, with the
invocation shown above sizesProcessorArray[0] = 5,
sizesProcessorArray[1] = 4, and sizesProcessorArray[2] = 12.
\par
The variable 'numberOfProcessors' contains the product
of all elements of sizesProcessorArray[], and can be seen as the
size of the one-dimensional processor array.
\par
Possible distribution specifications for a dimension are:
\begin{enumerate}
\item \verb'block' $n$. The elements in this shape dimension are distributed
  in groups of $n$ elements over the processor dimension.
Thus, for a block size $m$ and $n_p$ processors, the owner function is:
\[ \fnname{owner}( i ) = \floor{\frac{i}{m}} \mymod n_p \]

\item \verb'block'. Equivalent to 'blockcyclic' $\lceil n_e/n_p \rceil$,
   where $n_e$ is the
   number of elements in this shape dimension, and $n_p$ is the number
   of processors in this processor dimension. This is not only a handy
   shortcut, but it also allows more efficient code generation.

\item \verb'cyclic'. Equivalent to 'blockcyclic 1'. This is not only a handy
  shortcut, but it also allows more efficient code generation.

\item \verb'*'. All elements in this shape dimension are replicated on
  all processors in this dimension.

\item \verb'+'. Skip this shape dimension. This dimension is said to
      be {\em collapsed}.

\item \verb'-'. This distribution will never be used. This specification
      is used in formal parameters.
\end{enumerate}

Each distribution specification, except the 'collapsed' specification,
uses one processor dimension. The entire list of distributions must use
either the entire more-dimensional processor array, or only the
one-dimensional processor array. In other words, in a program with a
processor array of rank $n$, every distribution must have either 1
or $n$ non-collapsed dimensions.
\subsection{Pragmas}
\begin{verbatim}
pragma:
    identifier
|
    identifier "=" expression

pragmalist:
    /* empty */
|
    pragma
|
    pragmalist "," pragma

pragmas:
     "<<" pragmalist ">>"

opt_pragmas:
    /* empty */
|
    pragmas

\end{verbatim}
Several {\langnm} language constructs can be annotated with \defn{pragmas}.
These pragmas are intended to convey hints to the compilation system, and
do not (or at least should not) influence the behavior of the program.
\par
A pragma can either be a flag, whose mere presence is sufficient, or it
can have an arbitrary expression as value. For the list of supported pragmas,
please consult seperate documentation.
\par
At present, the following language constructs can have a pragma list:
expression, location, statement and declaration. If the need arises,
other languages constucts may be annotated too.
Examples:
\begin{verbatim}
<<const>>
<<maxsize=400>>
<<miniterations=0,maxiterations=2*n>>
\end{verbatim}

\section{Meaning of identifiers}
Identifiers, or names, refer to a variety of things: functions and procedures,
views, shapes and scalars.
\subsection{Basic types}
\label{s.types}
\begin{verbatim}
basetype:
    "string" | "boolean" | "natural"
|
    "integer" | "real" | "complex"

type: basetype
\end{verbatim}
The types {\tt natural}, {\tt integer}, {\tt real}, and {\tt complex}
represent a subset of the sets $\cal N$, $\cal Z$, $\cal R$,
and $\cal C$, respectively. In the following table the minimal range
of each type is specified. A specific implementation may broaden this
range.
\begin{center}
\begin{tabular}{ll}
type & range \\
\hline
{\tt natural} & $0 \ldots 32000$ \\
{\tt integer} & $-32000 \ldots 32000$ \\
{\tt real} & $\verb"-1e30" \ldots \verb"-1e-30" \cup \verb"1e-30" \ldots \verb"1e30"$ \\
\end{tabular}
\end{center}
{\tt Complex} numbers are represented as a pair of {\tt real} numbers
denoting real and imaginary part.
\par
The {\tt boolean} type represents the set $\{ \verb"true", \verb"false" \}$.
\par
Examples: \verb"boolean", \verb"INTEGER", \verb"Complex".
\subsection{Shapes}
\label{s.shapetype}
\begin{verbatim}
size: expression | "*"

size-list:
    <empty>
|
    size
|
    size-list commas size

sizes: "[" size-list opt_commas "]"

type: "shape" sizes distributions "of" basetype
\end{verbatim}
Shapes are similar to multi-dimensional arrays in other languages.
They are specified by the type of the elements and the extent in each
dimension. Every dimension is assumed to start at 0,
so only the upper bound needs to be specified.
\par
In a formal-parameter specification, a \verb"*" matches an arbitrary size
of the dimension, without binding it to a variable.
\par
Shapes of scalar types can be distributed, see section \ref{s.distributions}
for a description of distribution specifications.
If a distribution is given, a shape must have a distribution specification
for each of its dimensions.
Thus, the size of the distribution list must be the same as the the rank
of the shape. If no distribution is specified, a shape is replicated.
\par
Examples:
\begin{verbatim}
shape [20] of integer
Shape [10,20] of integer
\end{verbatim}
The following are only allowed in formal-parameter specifications:
\begin{verbatim}
shape [*,*] of complex
shape [*,2] of real
\end{verbatim}

\subsection{Views}
\begin{verbatim}
rank: sizes | INTEGER

type: "view" rank "of" basetype
\end{verbatim}
Views are functions that map index expressions onto shape elements.
Each view has a fixed number of dimensions and a fixed element type.
The mapping function and the extent in each dimension are determined
at run time.
\par
The number of dimensions (the \defn{rank}) of a view is specified as either
a number, or as a sequence of forms. The view declaration may
put restrictions on the size of a view dimension. If so, this
restriction will be enforced in subsequent view statements.
\par
Examples of valid view types are:
\begin{verbatim}
view 3 of integer
VIEW [*,*,*] of complex
VIEW [*,*] of complex
view [3,*] of integer
\end{verbatim}
The first two types are equivalent.
\par
The following are only allowed in a formal-parameter specification:
\begin{verbatim}
view [*,*] of integer
view [*,*] of natural
\end{verbatim}

\begin{remark}
The restrictions on view dimensions are at the moment not enforced.
\end{remark}

\section{Conversions}
Some operators may, depending on their operands, cause conversion of
the value from one type to another.
\begin{remark}
Yes, more details on this are needed, but we must first determine what
we want.
\end{remark}

\section{Expressions}
\subsection{Primary Expressions}
\begin{verbatim}
expression:
    <real>
|
    <integer>
|
    <string>
|
    "(" expression ")"
\end{verbatim}
Primary expressions are constants or expressions in parentheses.
\par
Examples: \verb"11", \verb'"Test"', \verb"(a+b)".
\subsection{Shape and view references}
\begin{verbatim}
selector: expression

selectorlist:
    <empty>
|
    selector
|
    selectorlist commas selector

selectors: "[" selectorlist "]" | <empty>

selection: identifier selectors

expression: selection
\end{verbatim}
Elements of shapes and views are referenced by selection expression.
Only single elements may be selected; therefore the size of the selector
list must be equal to the rank of the shape or view.
\par
References to scalars are considered selections with an empty selector list.
\par
Examples:
\verb"a[12]", \verb"b[1,2,3,4]", \verb"c".
\subsection{Function calls}
\begin{verbatim}
expressionlist:
    <empty>
|
    expression
|
    expressionlist commas expression

expressions: "(" expressionlist ")"

functioncall: identifier expressions

expression: functioncall
\end{verbatim}
A function call consists of a function name followed by 
parentheses containing a possibly empty, comma-separated list of expressions.
These expressions are called the \defn{actual parameters} of the function.
The size of the actual parameter list must be equal to the size of the
formal parameter list.
For details of the parameter passing mechanism, see Section \ref{s.fndecl}.
\subsection{Unary operators}
\begin{verbatim}
unary_operator: "not" | "+" | "-" | "\"

expression: unary_operator expression
\end{verbatim}
The following unary operators are supported: \verb"+" (unary plus),
\verb"-" (negation), {\tt not} (boolean inverse) and \verb"\"
(monadic set difference).
Unary operators bind more strongly than binary operators.
Thus, \verb'-a + b' is equivalent with \verb'(-a) + b'.
\subsection{Pragma expressions}
\begin{verbatim}
expression: pragmas expression
\end{verbatim}
This adds a pragma to the following expression. It binds as strongly
as unary operators, and more strongly than binary operators.
Thus, \verb'<<real>> a + b' is equivalent with \verb'(<<real>> a) + b'.
\subsection{Binary operators}
\begin{verbatim}
expression: expression binary_operator expression
\end{verbatim}
The following binary operators are supported:
\begin{center}
\begin{tabular}{|l|l|}
\hline
operator & description \\
\hline
\verb"[]" & selection \\
\hline
\verb"*" & multiplication \\
\verb"/" & division \\
{\tt div} & integer division \\
{\tt mod} & integer modulus operation \\
\hline
\verb"+" & addition \\
\verb"-" & subtraction \\
\verb"|" & set union \\
\verb"&" & set intersection \\
\hline
\verb".." & range \\
\hline
\verb"<" & less \\
\verb"<=" & less or equal \\
\verb">=" & greater or equal \\
\verb">" & greater \\
\verb"=" & equal \\
\verb"<>" & not equal \\
\hline
{\tt and} & logical and \\
{\tt or} & logical or \\
\hline
\end{tabular}
\end{center}
Each separator in this table starts a new, lower, level of operator precedence.
All operators are left-binding.
Thus \verb"a-b-c" is equivalent with \verb"a-(b-c)", and 
\verb"a+b*c" is equivalent with \verb"a+(b*c)".
\subsection{Distribution expressions}
\begin{verbatim}
expression:
    "blocksize" '(' location ')'
|
    "isowner" '(' location ',' expression ')'
|
    "sender" '(' location ')'
|
    "sender" '(' location ')'
\end{verbatim}
To access the distribution parameters of a shape, two special reference
expressions have been introduced. See Section \ref{s.shapetype} for an
explanation of a block size.
The {\tt identifier} must refer to a shape, and the {\tt selection} must
select a single element of a shape.
Views are not distributed, and therefore these expressions are meaningless
for views.
\par
Examples (assuming \verb"A" is a one-dimensional shape):
\verb"blocksize A", \verb"owner A[2]".
\subsection{Quantified selection expressions}
\begin{verbatim}
quantifiedselection: cardinalities "." selection

expression: quantifiedselection
\end{verbatim}
A quantified selection expression is an anonymous view.
It binds more strongly than selection.
\subsection{Complex number expressions}
\begin{verbatim}
expression: "complex" "(" expression "," expression ")"
\end{verbatim}
This constructs a complex number from a real and imaginary part,
respectively.
Examples: \verb"complex(1,0)", \verb"complex(42,22)".

\section{Shape and view inquiry expressions}
\begin{verbatim}
expression:
    "getsize" "(" location "," expression ")"
\end{verbatim}
This returns the size of a shape or view in a given dimension.
Example: \verb"getsize( A, 0 )".

\section{Declarations}
Except for cardinality variables, all identifiers in a
{\langnm} program must be declared.
Declarations can have \defn{global scope} (valid throughout the
program), or \defn{local scope} (valid within a function or
a procedure).
\subsection{Variable declarations}
\begin{verbatim}
variable:
     opt_pragmas identifier ":" type ";"

declaration:
    variable
\end{verbatim}
\par
Example:
\begin{verbatim}
a: shape [10,10] of natural;
\end{verbatim}
\subsection{Function and procedure declarations}
\label{s.fndecl}
\begin{verbatim}
passing-mechanism: <empty> | "ref"

argument: passing-mechanism identifier ":" opt_pragmas type

argument-list:
    <empty>
|
    argument
|
    argument-list commas argument

arguments: "(" argument-list opt_commas ")"

function:
    "function"
        identifier arguments
        "returns" identifier ":" type
        statementblock

procedure:
    opt_pragmas "procedure" identifier arguments statementblock

declaration:
    procedure
|
    function
\end{verbatim}
Functions and procedures may only be declared globally. 
The argument list defines a number of formal parameters.
If the flow type is \verb"in", a local copy of the actual parameter
will be made, and any changes to the parameter will not be visible outside
the function or procedure.
If the flow type is \verb"out", the parameter has an undefined value
upon entry, and all changes to the variable will be visible outside
the function or procedure.
If the flow type is \verb"inout", upon entry of the function the parameter
will have the value of the actual parameter. Any changes to the parameter
will be visible outside the function or procedure.
\par
Upon entry of a function, the value of the return variable is undefined.
During execution of the function the return variable may be assigned to
any number of times.
The value that is returned from a function is the value of the return
variable upon exit of the function.
\par
An example:
\begin{verbatim}
function Fac( n: natural ) returns m : natural
{
    m := 1;
    while (n > 1){
        m := m*n;
        n := n-1;
    }
}
\end{verbatim}
\subsection{External declarations}
\begin{verbatim}
declaration:
    opt_pragmas external_function
|
    opt_pragmas external_procedure
|
    opt_pragmas external_variable

external_function:
    "external" "function" identifier arguments
        "returns" identifier ":" type ";"
|
    "external" "function" identifier arguments "returns" type ";"

external_procedure:
    "external" "procedure" identifier arguments ";"

external_variable:
    "external" identifier ":" type ";"
\end{verbatim}
Since {\langnm} assumes a standard library for its support functions,
it allows external declarations. 

Examples:
\begin{verbatim}
external vnus_writes( in f: natural, in s: string );
external sin( in x: real ) returns real;
external cos( in x: real ) returns cosx : real;
\end{verbatim}

\section{Statements}
\subsection{Labels}
\begin{verbatim}
handle: identifier

statement: handle "::" action | action

action:
    empty
|
    control
|
    communication
|
    assignment
|
    view
|
    procedurecall

empty: ";"
\end{verbatim}
Each statement may be preceded by a label, consisting of a name
followed by two colons.
Example:
\begin{verbatim}
statement_one::  a := a+1;
\end{verbatim}
\subsection{Compound statements}
\begin{verbatim}
statementlist:
    statement
|
    statementlist statement

statementblock:
    "{" [declarationlist] [statementlist] "}"
\end{verbatim}
A \nonterm{statementblock} can be used to introduce a local scope with
local variables.
\subsection{Flow control statements}
\begin{verbatim}
cardinality:
    identifier ":" expression
|
    identifier expression
|
    identifier 
|
    "(" cardinality ")"

cardinalitylist:
    <empty>
|
    cardinality
|
    cardinalitylist commas cardinality

cardinalities: "[" cardinalitylist opt_commas "]"

while: "while" expression statementblock

iteration: "for" cardinalities statementblock

forall: "forall" cardinalities statementblock

forkall: "forkall" cardinalities statementblock

if:
    "if" expression statementblock "else" statementblock
|
    "if" expression statementblock

block:
    statementblock

control: while | iteration | forkall | forall | if | block
\end{verbatim}
The \verb'while' and \verb'if' statements have the usual semantics.
Each cardinality \verb'<v>:<r>' specifies an iteration variable
\verb'<v>' and an iteration range $0 \ldots \verb'<r>'-1$.
The \verb'for' statement iterates over its iteration variables in a fixed
order: each iterator iterates over its range in increasing order, and
each iterator nests the iterators to the right of it.
The \verb'forall' statement iterates over all possible combinations of
iterator values in arbitrary order.
The \verb'forkall' statement spawns a different thread for
each combination of values of the iterators.
These threads communicate with each other through the communication
statements.
For the moment, the compiler allows the \verb'forkall' statement only
in programs of a special format: programs that contain a single
forkall statement with one cardinality variable as the top-level statement
list are translated to a threaded execution program.
\begin{remark}
It is not clear whether the programmer should be responsible for
avoiding non-determinism, or the compiler.
We should discuss shared-memory implementations, and distributed-memory
implementations.
\end{remark}
\subsection{Communication statements}
\begin{verbatim}
communication: send | receive | barrier | broadcast

send: "send" "(" expression "," expression "," expression ")" ";"

receive: "receive" "(" expression "," expression "," expression ")" ";"

broadcast: "broadcast" "(" expression "," expression ")" ";"

barrier: "barrier" ";"
\end{verbatim}
Given the local processor number, the destination processor number and
an expression in that order, a \verb'send'
statement sends the result of the expression evaluation
to the given processor.
\par
Given the local processor number, the source processor number and a
variable expression in that order, a \verb'receive'
statement receives the contents of that
variable from the given processor.
\par
Given the local processor number and a variable,
 a \verb'broadcast' statement sends the result
of the expression evaluation to all processors in the system, including
the local processor. Every processor must receive the elment with the
\verb'receive' statement.
\par
The barrier statement provides synchronization between execution
threads as created by \verb'forkall'.
Each thread that enters a barrier waits until all other threads have
arrived at the barrier. At that moment all threads resume execution.
\par
For example, the following code implements the code
\begin{verbatim}
barrier;
forall [i:20] {
    A[19-i] := B[i];
}
barrier;
\end{verbatim}
but using explicit element-wise communication:
\begin{verbatim}
barrier;
forall [i:20] {
    desti := 19-i;
    if( sender( B[i] ) = p ){
	send( owner( A[desti] ), B[i] );
    }
    if( isowner( A[desti], p ) ){
	receive( owner B[i],  A[desti] );
    }
}
barrier;
\end{verbatim}
\subsection{Assignment statements}
\begin{verbatim}
assignment: location ":=" expression
\end{verbatim}
Examples:
\begin{verbatim}
a := 1;
a[] := 1;
v[12,23,5] := v[12,23,5]+v[12,23,6];
\end{verbatim}
\subsection{View statements}
\begin{verbatim}
view: "view" location cardinalities "<-" selection ";"
\end{verbatim}
A view statement constructs a new view.
\par
The keyword \verb'view' is necessary to distinguish this statement from
assignment to view and shape elements; in particular from 0-dimensional
selection. For example, the statements
\begin{verbatim}
a[] <- x[];    // not allowed, ambiguous
a[] := x[];
\end{verbatim}
require a look-ahead of two tokens (to the \verb'<-' or \verb':=') to
disambiguate. This is not possible with the parser generator that we use.
\par
Examples:
\begin{verbatim}
view v[i:10,j:10,k:20] <- A[10-i,j,k];
view v[i] <- A[i,i,1];
\end{verbatim}
\subsection{Procedure call statements}
\begin{verbatim}
procedurecall: identifier expressions
\end{verbatim}
Examples:
\begin{verbatim}
WriteS(f,"Hello World");
p();
\end{verbatim}

\section{Declaration lists}
\begin{verbatim}
declarationlist:
    declaration
|
    declarationlist statement_sep declaration

declaration: function | procedure | variable
\end{verbatim}
Declaration lists contain local or global declarations.
\par
Example:
\begin{verbatim}
i: integer;
v: view [*] of natural;
a: shape [20] of natural;
\end{verbatim}

\section{Programs}
\begin{verbatim}
opt_globalpragmas:
    <empty>
|
    "globalpragmas" pragmas statement_terminator

program: opt_globalpragmas [declarations] statements
\end{verbatim}
Example (\verb"vnusout" is a pre-defined variable that denotes
standard output, similarly, \verb"vnusin" is a pre-defined variable that
denotes standard input):
\begin{verbatim}
// fac.sv -- The Factorial Function.

globalpragmas <<realshape>>;

external procedure vnusstio_writes( f: natural, s: string );
external procedure vnusstio_writen( f: natural, n: natural );
external procedure vnusstio_readn( f: natural, ref n: natural );
external vnusout: natural;
external vnusin: natural;

x : natural;

// Iterative factorial function.
function Fac( n: natural ) returns m: natural 
{
    m := 1;
    while( n > 1 ){
        m := m*n;
        n := n-1;
    }
}

/* Recursive factorial function. */
function FacRec( n: natural ) returns m: natural 
{
    if( n = 0 ){
        m := 1;
    }
    else {
        m := n * FacRec(n-1);
    }
}

vnusstio_readn( vnusin, x );
while( x <> 1000 ){
    vnusstio_writes(vnusout, "\nIterative:\t");
    vnusstio_writen(vnusout, x);
    vnusstio_writes(vnusout, "! = ");
    vnusstio_writen(vnusout, Fac(x));
    vnusstio_writes(vnusout, "\nRecursive:\t");
    vnusstio_writen(vnusout, x);
    vnusstio_writes(vnusout, "! = ");
    vnusstio_writen(vnusout, FacRec(x));
    vnusstio_writes(vnusout, "\n\n");
    vnusstio_writes(vnusout, "integer (1000 to stop): ");
    vnusstio_readn(vnusin, x);
}
vnusstio_writes( vnusout, "\nHave a nice day!\n" );
\end{verbatim}
\end{document}
